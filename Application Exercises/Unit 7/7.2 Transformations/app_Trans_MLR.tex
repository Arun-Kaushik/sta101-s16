\documentclass[12pt]{article}
%%%%%%%%%%%%%%%%
% Packages
%%%%%%%%%%%%%%%%

\usepackage[top=1cm,bottom=1.25cm,left=1.25cm,right= 1.25cm]{geometry}
\usepackage[parfill]{parskip}
\usepackage{graphicx, fontspec, xcolor,multicol, enumitem, setspace, amsmath, changepage}
\DeclareGraphicsRule{.tif}{png}{.png}{`convert #1 `dirname #1`/`basename #1 .tif`.png}

%%%%%%%%%%%%%%%%
% User defined colors
%%%%%%%%%%%%%%%%

% Pantone 2015 Fall colors
% http://iwork3.us/2015/02/18/pantone-2015-fall-fashion-report/
% update each semester or year

\xdefinecolor{custom_blue}{rgb}{0, 0.32, 0.48} % FROM SPRING 2016 COLOR PREVIEW
\xdefinecolor{custom_darkBlue}{rgb}{0.20, 0.20, 0.39} % Reflecting Pond  
\xdefinecolor{custom_orange}{rgb}{0.96, 0.57, 0.42} % Cadmium Orange
\xdefinecolor{custom_green}{rgb}{0, 0.47, 0.52} % Biscay Bay
\xdefinecolor{custom_red}{rgb}{0.58, 0.32, 0.32} % Marsala

\xdefinecolor{custom_lightGray}{rgb}{0.78, 0.80, 0.80} % Glacier Gray
\xdefinecolor{custom_darkGray}{rgb}{0.35, 0.39, 0.43} % Stormy Weather

%%%%%%%%%%%%%%%%
% Color text commands
%%%%%%%%%%%%%%%%

%orange
\newcommand{\orange}[1]{\textit{\textcolor{custom_orange}{#1}}}

% yellow
\newcommand{\yellow}[1]{\textit{\textcolor{yellow}{#1}}}

% blue
\newcommand{\blue}[1]{\textit{\textcolor{blue}{#1}}}

% green
\newcommand{\green}[1]{\textit{\textcolor{custom_green}{#1}}}

% red
\newcommand{\red}[1]{\textit{\textcolor{custom_red}{#1}}}

%%%%%%%%%%%%%%%%
% Coloring titles, links, etc.
%%%%%%%%%%%%%%%%

\usepackage{titlesec}
\titleformat{\section}
{\color{custom_blue}\normalfont\Large\bfseries}
{\color{custom_blue}\thesection}{1em}{}
\titleformat{\subsection}
{\color{custom_blue}\normalfont}
{\color{custom_blue}\thesubsection}{1em}{}

\newcommand{\ttl}[1]{ \textsc{{\LARGE \textbf{{\color{custom_blue} #1} } }}}

\newcommand{\tl}[1]{ \textsc{{\large \textbf{{\color{custom_blue} #1} } }}}

\usepackage[colorlinks=false,pdfborder={0 0 0},urlcolor= custom_orange,colorlinks=true,linkcolor= custom_orange, citecolor= custom_orange,backref=true]{hyperref}

%%%%%%%%%%%%%%%%
% Instructions box
%%%%%%%%%%%%%%%%

\newcommand{\inst}[1]{
\colorbox{custom_blue!20!white!50}{\parbox{\textwidth}{
	\vskip10pt
	\leftskip10pt \rightskip10pt
	#1
	\vskip10pt
}}
\vskip10pt
}
\usepackage{fancyvrb}	% for colored code chunks
\usepackage{lscape}

%%%%%%%%%%%
% App Ex number    %
%%%%%%%%%%%

% DON'T FORGET TO UPDATE

\newcommand{\appno}[1]
{7.2}

%%%%%%%%%%%%%%
% Turn on/off solutions       %
%%%%%%%%%%%%%%
%:

% Off
%\newcommand{\soln}[2]{$\:$\\ \vspace{#1}}{}

% On
\newcommand{\soln}[2]{\textit{\textcolor{custom_red}{#2}}}{}

%%%%%%%%%%%%%%%%
% Document
%%%%%%%%%%%%%%%%

\begin{document}
\fontspec[Ligatures=TeX]{Helvetica Neue Light}

Dr. \c{C}etinkaya-Rundel \hfill Sta 101: Data Analysis and Statistical Inference \\
Duke University - Department of Statistical Science \hfill \\

\ttl{Application exercise \appno{}: \\
Interpreting models with a transformed response}

\inst{$\:$ \\
Team name: \rule{10cm}{0.5pt} \\
$\:$ \\
Lab section: $\qquad$ 8:30 $\qquad$ 10:05 $\qquad$ 11:45 $\qquad$ 1:25 $\qquad$ 3:05 $\qquad$ 4:40 \\
$\:$ \\
Write your responses in the spaces provided below. WRITE LEGIBLY and SHOW ALL WORK! 
Only one submission per team is required. One team will be randomly selected and their 
responses will be discussed and graded. Concise and coherent are best!}

%%%%%%%%%%%%%%%%%%%%%%%%%%%%%%%%%%%%

\section*{Predicting income in the US}

Each year since 2005, the US Census Bureau surveys about 3.5 million households with The American Community Survey (ACS). Data collected 
from the ACS have been crucial in government and policy decisions, helping to determine the allocation of more than \$400 billion in federal and 
state funds each year. For example, funds for the Adult Education and Family Literacy Act are distributed to states taking into consideration data 
from the ACS on number of adults 16 and over without a high school diploma. This act is the primary source of federal funding for adults with low 
basic skills seeking further education or English language services, and Department of Education uses ACS data to ensure the efficient distribute funds.

In this application exercise we will analyze data from the ACS, and use the fact that it is ``a random survey" to make inferences about the US 
population at large.

List of variables:
\begin{enumerate}
\item \texttt{income}: Yearly income (wages and salaries)
\item \texttt{employment}: Employment status, not in labor force, unemployed, or employed
\item \texttt{hrs\_work}: Weekly hours worked
\item \texttt{race}: Race, White, Black, Asian, or other
\item \texttt{age}: Age
\item \texttt{gender}: Male or female
\item \texttt{citizens}: Whether respondent is a US citizen or not
\item \texttt{time\_to\_work}: Travel time to work
\item \texttt{lang}: Language spoken at home, English or other
\item \texttt{married}: Whether respondent is married or not
\item \texttt{edu}: Education level, hs or lower, college, or grad
\item \texttt{disability}: Whether respondent is disabled or not
\item \texttt{birth\_qrtr}: Quarter in which respondent is born, Jan thru Mar, Apr thru Jun, Jul thru Sep, or Oct thru Dec 
\end{enumerate}

First, load the R Markdown template:

{\footnotesize
\begin{Verbatim}[frame=single, formatcom=\color{blue}]
download("http://stat.duke.edu/courses/Spring16/sta101.001/rmd/app_Trans_MLR.Rmd", 
  destfile = "app_Trans_MLR.Rmd")
\end{Verbatim}
}

%

\begin{enumerate}

\item  Load data, and subset for those who were employed:

\begin{Verbatim}[frame=single, formatcom=\color{blue}]
load(url("http://stat.duke.edu/~mc301/data/acs.RData"))

acs_emp <- acs %>%
  filter(employment == "employed", income > 0)
\end{Verbatim}

Before you proceed, confirm that this leaves you with 787 observations.

\item Suppose we only want to consider the following explanatory variables: \texttt{hrs\_work}, \texttt{race}, 
\texttt{age}, \texttt{gender}, \texttt{citizen}. Fit the full model using only the explanatory variables listed above, 
and report its adjusted $R^2$. Remember your response variable is \texttt{log(income)}, not \texttt{income}. 
Below is a neat trick for getting the just the adjusted $R^2$:

\begin{Verbatim}[frame=single, formatcom=\color{blue}]
m_full <- lm(log(income) ~ hrs_work + race + age + gender + citizen, 
  data = acs_emp)

summary(m_full)$adj.r.squared
\end{Verbatim}

\item Conduct model selection using the backwards adjusted $R^2$ method, and report the adjusted 
$R^2$ for the final model.

\item Check diagnostics for your final model using appropriate plots. \textit{Hint:} Code from today's 
slides should be helpful.

\item Interpret the slope for one numerical and one categorical predictor. \textit{Hint:} To exponentiate a 
value in \texttt{R}, use the \texttt{exp()} function, e.g. to calculate $e^3$, use

\begin{Verbatim}[frame=single, formatcom=\color{blue}]
exp(3)
\end{Verbatim}

\end{enumerate}

%


\end{document}