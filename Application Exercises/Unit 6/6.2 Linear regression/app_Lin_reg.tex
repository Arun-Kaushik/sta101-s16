\documentclass[11pt]{article}

%%%%%%%%%%%%%%%%
% Packages
%%%%%%%%%%%%%%%%

\usepackage[top=1cm,bottom=1.25cm,left=1.25cm,right= 1.25cm]{geometry}
\usepackage[parfill]{parskip}
\usepackage{graphicx, fontspec, xcolor,multicol, enumitem, setspace, amsmath, changepage}
\DeclareGraphicsRule{.tif}{png}{.png}{`convert #1 `dirname #1`/`basename #1 .tif`.png}

%%%%%%%%%%%%%%%%
% User defined colors
%%%%%%%%%%%%%%%%

% Pantone 2015 Fall colors
% http://iwork3.us/2015/02/18/pantone-2015-fall-fashion-report/
% update each semester or year

\xdefinecolor{custom_blue}{rgb}{0, 0.32, 0.48} % FROM SPRING 2016 COLOR PREVIEW
\xdefinecolor{custom_darkBlue}{rgb}{0.20, 0.20, 0.39} % Reflecting Pond  
\xdefinecolor{custom_orange}{rgb}{0.96, 0.57, 0.42} % Cadmium Orange
\xdefinecolor{custom_green}{rgb}{0, 0.47, 0.52} % Biscay Bay
\xdefinecolor{custom_red}{rgb}{0.58, 0.32, 0.32} % Marsala

\xdefinecolor{custom_lightGray}{rgb}{0.78, 0.80, 0.80} % Glacier Gray
\xdefinecolor{custom_darkGray}{rgb}{0.35, 0.39, 0.43} % Stormy Weather

%%%%%%%%%%%%%%%%
% Color text commands
%%%%%%%%%%%%%%%%

%orange
\newcommand{\orange}[1]{\textit{\textcolor{custom_orange}{#1}}}

% yellow
\newcommand{\yellow}[1]{\textit{\textcolor{yellow}{#1}}}

% blue
\newcommand{\blue}[1]{\textit{\textcolor{blue}{#1}}}

% green
\newcommand{\green}[1]{\textit{\textcolor{custom_green}{#1}}}

% red
\newcommand{\red}[1]{\textit{\textcolor{custom_red}{#1}}}

%%%%%%%%%%%%%%%%
% Coloring titles, links, etc.
%%%%%%%%%%%%%%%%

\usepackage{titlesec}
\titleformat{\section}
{\color{custom_blue}\normalfont\Large\bfseries}
{\color{custom_blue}\thesection}{1em}{}
\titleformat{\subsection}
{\color{custom_blue}\normalfont}
{\color{custom_blue}\thesubsection}{1em}{}

\newcommand{\ttl}[1]{ \textsc{{\LARGE \textbf{{\color{custom_blue} #1} } }}}

\newcommand{\tl}[1]{ \textsc{{\large \textbf{{\color{custom_blue} #1} } }}}

\usepackage[colorlinks=false,pdfborder={0 0 0},urlcolor= custom_orange,colorlinks=true,linkcolor= custom_orange, citecolor= custom_orange,backref=true]{hyperref}

%%%%%%%%%%%%%%%%
% Instructions box
%%%%%%%%%%%%%%%%

\newcommand{\inst}[1]{
\colorbox{custom_blue!20!white!50}{\parbox{\textwidth}{
	\vskip10pt
	\leftskip10pt \rightskip10pt
	#1
	\vskip10pt
}}
\vskip10pt
}
\usepackage{fancyvrb}	% for colored code chunks

%%%%%%%%%%%
% App Ex number    %
%%%%%%%%%%%

% DON'T FORGET TO UPDATE

\newcommand{\appno}[1]
{6.2}

%%%%%%%%%%%%%%
% Turn on/off solutions       %
%%%%%%%%%%%%%%

% Off
\newcommand{\soln}[2]{$\:$\\ \vspace{#1}}{}

%%% On
%\newcommand{\soln}[2]{\textit{\textcolor{custom_red}{#2}}}{}

%%%%%%%%%%%%%%%%
% Document
%%%%%%%%%%%%%%%%

\begin{document}
\fontspec[Ligatures=TeX]{Helvetica Neue Light}

Dr. \c{C}etinkaya-Rundel \hfill Sta 101: Data Analysis and Statistical Inference \\
Duke University - Department of Statistical Science \hfill \\

\ttl{Application exercise \appno{}: \\
Linear regression}

\inst{$\:$ \\
Team name: \rule{10cm}{0.5pt} \\
$\:$ \\
Lab section: $\qquad$ 8:30 $\qquad$ 10:05 $\qquad$ 11:45 $\qquad$ 1:25 $\qquad$ 3:05 $\qquad$ 4:40 \\
$\:$ \\
Write your responses in the spaces provided below. WRITE LEGIBLY and SHOW ALL WORK! 
Only one submission per team is required. One team will be randomly selected and their 
responses will be discussed and graded. Concise and coherent are best!}

%%%%%%%%%%%%%%%%%%%%%%%%%%%%%%%%%%%%

\section*{Driver age and highway sign-reading distance}

In a study of the legibility and visibility of highway signs, a Pennsylvania research firm determined the maximum distance at which 
each of thirty drivers could read a newly designed sign. The thirty participants in the study ranged in age from 18 to 82 years old. 
The government agency that funded the research hoped to improve highway safety for older drivers, and wanted to examine the 
relationship between age and the sign legibility distance.

First, load the dataset and the R Markdown template:

{\scriptsize
\begin{Verbatim}[frame=single, formatcom=\color{blue}]
download.file("https://stat.duke.edu/courses/Spring16/sta101.001/rmd/app_Lin_reg.Rmd", destfile = "app_Lin_reg.Rmd")
\end{Verbatim}
}

And make sure the dataset is loaded as well:

{\scriptsize
\begin{Verbatim}[frame=single, formatcom=\color{blue}]
vision <- read.csv("https://stat.duke.edu/~mc301/data/vision.csv")
\end{Verbatim}
}

If you have questions about the \texttt{R} syntax, refer to the last lab and the slides, or just ask. Upload the R Markdown and HTML 
files to Sakai under the appropriate assignment. Note that the R Markdown file includes additional information and helper code that 
will help you.

\textbf{Extremely important:} Note that the R chunks are currently turned off with 
\texttt{eval = FALSE}, since the code is incomplete (you will be filling in the 
appropriate code). Once you do, turn on the chunk by setting \texttt{eval = TRUE}.


\subsection*{Part 1 - Finish in class}

\begin{enumerate}
\item Fit a linear model predicting distance at which drivers can read highway signs (in feet) based on age (in years). Save this model 
as \texttt{mod}, include the regression output in your answer, and write the linear model.

\item Interpret the slope and the intercept in context of the data and the model.

\item Interpret the p-value for the slope in context of the data  and the model.

\item Based on this p-value, does age appear to be a significant predictor of distance at which drivers can read highway signs? Make sure 
to state the hypotheses and the significance level you are using.

\item Construct a 95\% confidence interval for the slope and interpret it.

\item Predict the maximum distance at which a 30 year old driver can read highway signs, and report this prediction with a 95\%
prediction interval.

\end{enumerate}

\vfill

\begin{center}
\textit{See next page for Part 2}
\end{center}

%

\subsection*{Part 2 - Work on it in class if there is extra time, or on your own after class}

\begin{enumerate}

\item Confirm the values of the slope and the intercept using summary statistics of the data, i.e. means and standard deviations of the 
variables as well as the correlation between them.  Note that we have provided some of the code for you to calculate these summary statistics, and placed them in a data frame called \texttt{vision\_summ}, you just need to use these values in the next chunk to calculate the slope and the intercept.

\item Confirm the t-score given on the regression output for the slope, i.e. show how it can be calculated using other values from the output.

\item Calculate the $R^2$ using the ANOVA output for this model. Confirm that this value matches the value of $R^2$ reported 
in the regression output.

\end{enumerate}

\end{document}